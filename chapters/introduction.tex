%%%%%%%%%%%%%%%%%%%%%%%%%%%%%%%%%%%%%%%%%%%%%%%%%%%%%%%%%%%%%%%%%%%%%%%%%%%
\chapter{Introduction}
%%%%%%%%%%%%%%%%%%%%%%%%%%%%%%%%%%%%%%%%%%%%%%%%%%%%%%%%%%%%%%%%%%%%%%%%%%%
Human Resource Management (HRM) plays an important role in the success and efficiency of institutions. HRM is responsible for managing the most valuable asset of any organization—its people. The primary functions of HRM include recruitment, training, development, performance management, and employee relations. These functions are essential for building a productive and thriving workforce, which is crucial for achieving organizational goals and maintaining a competitive edge in the market \cite{a12019}\cite{t12020}.

HR professionals act as strategic partners to the management team, advising on managing human resources effectively to meet the institution's current and future needs. They are involved in developing strategies for talent acquisition, employee retention, and skill development, ensuring the workforce is aligned with the organization's objectives. Additionally, HRM plays a significant role in fostering a positive work environment, enhancing employee engagement, and building loyalty and commitment among employees \cite{b12021}\cite{d12022}.

A Human Resources Information System (HRIS) is a software solution designed to manage and automate core HR processes. HRIS serves as a centralized repository for employee data, facilitating efficient data management, payroll processing, benefits administration, time and attendance tracking, and performance management \cite{m22024}\cite{o12023}. The implementation of an HRIS transforms HR from an administrative function to a strategic one, enabling HR professionals to focus on more value-added activities such as strategic planning and decision-making \cite{n12023}.

By automating routine HR tasks, HRIS reduces the administrative burden on HR professionals, allowing them to allocate more time to strategic initiatives. This automation leads to increased efficiency, accuracy, and compliance with labor laws and regulations. Moreover, HRIS provides real-time data access and advanced analytics capabilities, which are essential for informed decision-making and strategic workforce planning \cite{gym12024}\cite{h12022}.

HRIS also enhances the employee experience by offering self-service portals where employees can access and update their personal information, request time off, and manage their benefits. This not only improves employee satisfaction but also reduces the workload on HR staff. Furthermore, the integration of HRIS with other business systems ensures seamless data flow and consistency across the organization, contributing to overall operational efficiency \cite{a22023}\cite{s12022}.

\section{Project Context}
    
    The Human Resource and Management Office (HRMO) at Ateneo de Naga University has been utilizing a system that has served the institution well over the years. This system, consisting of disparate databases and applications, has been instrumental in managing HR processes and maintaining employee records. However, as the university has grown, the limitations of this system have become more apparent. The separation of the HRIS schema from the University Integrated Information Systems (UIIS) has led to challenges such as data duplication and the need for more streamlined processes. These issues can hinder the effectiveness of HR operations, making it difficult to manage the increasing volume of data and integrate with other university systems.
    
    To address these challenges and support the university's continued growth, there is a need to develop and implement a centralized HRIS. This new system will ensure data consistency, accuracy, and accessibility, thereby streamlining HRMO processes and improving overall data management. The centralized HRIS will also facilitate better integration with other university systems, enhancing the institution's ability to manage HR data comprehensively and efficiently. The new HRIS will be designed to interact seamlessly with various stakeholders within the university. HRMO personnel will be responsible for data entry, managing employee records, and processing HR-related tasks. University management will access HR data for decision-making and strategic planning. Other departments requiring HR data for various administrative and operational purposes will also benefit from the new system.

    The study will involve key stakeholders who will provide valuable insights and feedback throughout the project. HRMO personnel, with their firsthand experience with the current system, will help identify areas for improvement and define requirements for the new HRIS. MIS office personnel, with their technical expertise, will be crucial in designing and implementing the new system. University management will ensure that the new HRIS aligns with the institution's overall goals and objectives. The new HRIS will handle various types of data to support HR operations. Input data will include employee records, attendance data, leave applications, and other HR-related information. Output data will consist of reports on employee information, attendance, leave status, and other HR metrics. These reports will be essential for decision-making, compliance, and strategic planning.
    
\section{Purpose and Description}

    The main purpose of the project is to redesign, develop and implement a centralized Human Resource Information System (HRIS) to enhance data management, streamline HR processes, and support the university's growth. This initiative aims to address the limitations of the current system by providing a more integrated and efficient solution that meets the evolving needs of the institution.

    The project will begin with an information-gathering and system analysis phase. This initial phase will involve collecting detailed feedback from HRMO personnel, university management, and other stakeholders to understand the current system's limitations and requirements. Analyzing the existing system will help identify specific areas that need improvement and features that should be incorporated into the new HRIS. Following the analysis, the project will proceed with the design and development of the new HRIS. This includes creating modules for attendance monitoring, leave management, user authentication, and report generation. 
    
    The new system will be designed to accommodate the current requirements that the current system was unable to fully support. Additionally, the project will focus on redesigning HR workflows to make them more efficient and user-friendly. Integration with other university systems will be a key component, ensuring seamless data flow and accessibility across different departments. Finally, the project will involve migrating existing employee data to the new HRIS, ensuring that no valuable information is lost in the transition.

    The new HRIS will bring substantial benefits to various stakeholders within the university:
    
    \begin{itemize}
        \item[] \textbf{HRMO personnel.} The new HRIS will significantly improve data integrity and streamline HR processes, allowing HRMO staff to manage employee information more effectively. This will reduce the time spent on manual data entry and corrections, enabling staff to focus on more strategic HR activities.

        \item[] \textbf{University Management.} University management will benefit from enhanced reporting and analytics capabilities provided by the new HRIS. This will facilitate better decision-making and strategic planning by providing accurate and up-to-date HR data.

        \item[] \textbf{Employees.} Employees will experience more efficient HR services, such as faster processing of leave applications and more accurate attendance tracking. The user-friendly interface of the new HRIS will also make it easier for employees to access and update their personal information.

        \item[] \textbf{Departments.} Other departments that require HR data for various administrative and operational purposes will benefit from the improved data accessibility and integration with other university systems. This will ensure that all departments have access to consistent and accurate HR information.

        \item[] \textbf{The University.} The university will benefit from a scalable system that supports future growth and integration needs. The new HRIS will facilitate easier access, management, and reporting of employee data, thereby enhancing overall operational effectiveness. By addressing the limitations of the current system, the new HRIS will provide a robust foundation for the university's HR operations, supporting its mission and strategic goals.
    \end{itemize}
    
\section{Objectives}

    The main objective of the study is to redesign and implement the core Human Resource Information System (HRIS) functionalities for the Ateneo de Naga University. In order to achieve the main objective, the following objectives must be accomplished:

    \begin{enumerate}
        \item Collect and document data requirements to ensure they align with stakeholders' needs and project goals.
        \item Create design specifications for each system component and module, including data models and interfaces.
        \item Review and confirm the proposed system design with stakeholders to ensure it meets project requirements.
        \item Develop and implement the functionality of modules based on the defined design and requirements.

        \item Gather feedback from stakeholders on the functionality and usability of all modules.
        \item Review the completed work from each iteration and demonstrate the new features to stakeholders.
        \item Assess the system's functionality, performance, and user experience.
        \item Ensure the system meets the needs and expectations of the end-users.
        \item Conclude the testing phase and prepare test reports summarizing testing activities and results.
        \item Prepare the system and infrastructure for deployment to ensure a smooth transition to production.
        \item Deploy the system to the production environment.
        \item Develop comprehensive training materials and documentation to support end-user training.
        \item Finalize detailed technical specifications documenting the system architecture, design, and implementation details.
    \end{enumerate}
    
\section{Scope and Limitation}
    
    % The project is focused on redesigning the Human Resource Information System (HRIS) to address the limitations of the current system and to meet the evolving needs of the university. The project will primarily concentrate on HR processes such as enhancing employee information management, streamlining performance evaluations, improving training and development tracking, maintaining comprehensive health and safety records, automating administrative tasks like generating certificates of employment and employment contracts, monitoring employee status and assignments, managing academic awards and faculty ranks, and facilitating report generation, 

    The project is focused on redesigning the Human Resource Information System (HRIS) to address the limitations of the current system and to meet the evolving needs of the university. The project will primarily concentrate on HR processes such as enhancing employee information management, streamlining performance evaluations, improving training and development tracking, maintaining comprehensive health and safety records, automating administrative tasks like generating certificates of employment and employment contracts, monitoring employee status and assignments, managing academic awards and faculty ranks, and facilitating report generation. The project shall also include the migration from the current system to UIIS. This includes covering data migration of records from the previous schema to migrating to the new HR schema and transitioning from the current system's Database Management System -- MySQL to Oracle.
    
    Additionally, the system will include functionalities for managing staff attendance and work schedules, handling leave applications and leave credits, and managing faculty attendance and schedules in connection with the integration of ADNU-ONE which cover these modules. This ensures that the new system is robust, user-friendly, and capable of handling the university's current and future requirements. The project will begin with gathering information and analyzing the existing system to identify its limitations and areas for improvement. This analysis will inform the design and development of the new HRIS, which will include several key modules.
    
    The modules to be developed include:
    
    \begin{itemize}
        \item[] \textbf{User Authentication and Authorization Module.} This module is essential for providing secure access mechanisms to ensure that only authenticated users have appropriate permissions within the HRIS. It will utilize GBOX accounts for user authentication, which offers easy access and enhanced security, thereby improving the overall user experience. By integrating GBOX, the system ensures that users can seamlessly log in using their existing credentials, reducing the need for multiple passwords and enhancing security through centralized authentication.
        \item[] \textbf{User Privileges.} The User Privileges module will offer comprehensive administrative tools for managing users, base tables, and other system settings. This module is crucial for security purposes as it ensures that users have the appropriate access levels and permissions based on their roles within the organization. By carefully managing user privileges, the system can prevent unauthorized access to sensitive information and maintain data integrity.
        \item[] \textbf{HRIS Modules.} These modules will contains functionalities to manage HR processes such as:
        \begin{itemize}
            \item[] \textbf{Employee Information Management:}
            Storing and managing personal and professional information of employees.

            \item[] \textbf{Performance Evaluations: }
            Streamlining the process of evaluating employee performance to ensure accurate and timely assessments.

            \item[] \textbf{Training and Development Tracking: } Recording and tracking training sessions attended by employees to support their professional growth.
            
            \item[] \textbf{Health and Safety Records: } Maintaining comprehensive health records to ensure the well-being of employees and compliance with health regulations.
            
            \item[] \textbf{Administrative Tasks: } Generating certificates of employment and employment contracts to reduce administrative burden and improve accuracy.
            
            \item[] \textbf{Monitoring Employee Status and Assignments:  } Tracking status movements and job assignments of employees to ensure proper record-keeping and organizational structure.
            
            \item[] \textbf{Managing Academic Awards and Faculty Ranks: } Handling academic awards and faculty ranks to recognize and incentivize employee achievements.
        \end{itemize}


        \item[] \textbf{TIMESYS: Attendance Monitoring (Staff) Modules.} These modules will handle the monitoring and recording of staff attendance. They include functionalities such as capturing login and logout times, processing raw attendance data, managing work schedules, and computing overtime. By accurately tracking attendance, the system can ensure compliance with work schedules and provide reliable data for payroll and performance evaluations.
        \item[] \textbf{FACSYS: Attendance Monitoring (Faculty) Modules.} The FACSYS modules will manage the monitoring and recording of faculty attendance. This includes tracking attendance by semester and school year, managing faculty schedules, and processing attendance reports. These modules ensure that faculty attendance is accurately recorded and reported, facilitating compliance with academic schedules and workload requirements.
        \item[] \textbf{LEAVESYS: Leave Management.} This module will manage employees' leave applications and processing. It includes storing approved leave applications, tracking leave reasons and credits, and processing leave applications with leave credits reports. By automating leave management, the system can streamline the approval process, ensure accurate tracking of leave balances, and provide valuable data for decision-making.
        \item[] \textbf{Optimized Reports Generation.} The current system lacks optimization on their reports causing delays in report generation. With this, the project aims to allow users to set parameters for reports, generate report outputs in various formats such as CSV, Excel, or PDF, and export reports efficiently for faster further analysis and record-keeping. This functionality ensures that users can easily generate and access the data they need for strategic decision-making, compliance reporting, and operational analysis.
    \end{itemize}
    
    Given the wide scope of this project, the focus will be the modules mentioned, with room for scalability to accommodate additional features for the employee view in the future. The project does not include the development of new hardware infrastructure, as it will utilize existing university resources. Additionally, modules unrelated to HR processes, such as financial management or student information systems, are not within the scope of this project.
    
    By concentrating on these areas, the project aims to create a centralized HRIS that enhances data management, streamlines HR processes, and supports the university's growth. The new system will provide a scalable and integrated solution that addresses the limitations of the current system, ultimately benefiting all stakeholders involved.


