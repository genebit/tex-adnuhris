%%%%%%%%%%%%%%%%%%%%%%%%%%%%%%%%%%%%%%%%%%%%%%%%%%%%%%%%%%%%%%%%%%%%%%%%%%%
\chapter{Review of Related Systems and Literature}
%%%%%%%%%%%%%%%%%%%%%%%%%%%%%%%%%%%%%%%%%%%%%%%%%%%%%%%%%%%%%%%%%%%%%%%%%%%

This chapter provides the relevant literature that is essential for developing a better understanding of the problem of the study and the significance of its resolutions. Additionally, existing features and/or systems that can assist in resolving this issue are identified. Researchers found it beneficial to evaluate the gaps in existing systems as it guides in determining the features that a system should have to successfully solve the problem of the study.

% ------------------------------------------------------------------ %
% ------Review of Related Literature-------------------------------- %
% ------------------------------------------------------------------ %

\section{Review of Related Literature}
    Human Resource Information Systems (HRIS) has undergone significant transformation, driven by technological advancements and evolving organizational needs. This literature review provides a foundational understanding of HRIS and its historical progression over time from basic HR processes to complex systems that support a wide range of HR-related tasks. It emphasizes how essential HRIS is to improving corporate effectiveness, simplifying HR procedures, and enabling data-driven decision-making. It also includes the current trends as well as how the progression of technology influences the future of HRM systems.
    
    \subsection{Definition and Scope}
        Technological developments and the change of requirements of organizations on the effective management of their human resources have a significant impact on the evolution of the Human Resources Information System. The primary functions of the early HRIS are focused on payroll processing and employee record-keeping \cite{srr12023}. As time passes, these systems have grown into complex platforms which now include several HR duties such as hiring, training, performance management, and analytics \cite{ml12019}.
        
        The historical development of HRIS appears to date back more than six decades, where the emergence of mainframe systems that support the basic HR functions such as payroll processing and government reporting \cite{srr12023}. Organizations’ strategies in managing their human capital have begun to change as these early technologies paved the way for the integration of HR procedures onto digital platforms. As technology advances, the HR departments adopt advanced HRIS, which effectively improves data management, reporting,  and decision-making capabilities  \cite{ml12019}. The way that HRIS has evolved indicates a change towards the idea of utilizing technology to improve organization efficiency of organizations and streamline HR processes.
        
        The HRIS’s main components and functionality include a wide range of functionalities that are designed to support various aspects of HR administration. These often include modules for managing employee information management, payroll processing, benefits administration, recruitment, training, and performance evaluation \cite{ep12023}. The core of an HRIS is a centralized repository, which offers a secure and readily available platform for handling and storing personnel data \cite{h12020}. Furthermore, analytics tools are frequently integrated into modern HRIS systems, allowing businesses to get knowledge from HR data and decide on workforce planning and development \cite{aa12021}.
        
    \subsection{Importance of HRIS in Organizations}
    
        Human Resources Information Systems (HRIS) are valuable in organizations, offering numerous advantages to organizations and improving HR operations. HR operations are made more efficient by the use of HRIS, which play an essential part in simplifying everything from performance management to recruitment. HR departments may function more effectively and efficiently using HRIS by centralizing employee data and automating repetitive processes \cite{s12024}. Since it enables HR professionals to concentrate on strategic initiatives rather than administrative responsibilities, the integration of HRIS has become essential to the success of organizations, thereby improving the overall operational efficiency \cite{ta12023}. 

        In addition, the advantages of using HR Information Systems for organizations go beyond operational enhancements and extend to strategic advantages. By offering real-time insights into workforce dynamics and performance measures, HRIS enables data-driven decision-making \cite{hae12021}. Organizations can determine skill shortages, streamline their staff management plans, and match HR activities with corporate goals by utilizing HR's information. Furthermore, HRIS improves regulatory compliance and fosters HR process transparency, which in turn cultivates an organizational culture of responsibility and fairness \cite{pfsa12023}.
        
        It has been demonstrated that implementing HR systems can substantially improve an organization's efficiency. HRIS simplifies operations and lessens the administrative burden on HR professionals by centralizing HR data and automating manual procedures \cite{arc12020}. By integrating self-service substitutes for activities like leave management and performance assessments, this transition towards digital HR administration improves not just the accessibility and accuracy of data but also the entire experience of employees \cite{f12022}. Thus, implementing HRIS is a strategic investment that helps businesses maximize their HRM procedures and achieve superior results in terms of hiring, developing, and managing people \cite{aab12019}.
        

    \subsection{HR in Educational Institutions}
        The challenges associated with human resources in educational institutions are complex and demand considerable planning in order to guarantee efficient people management. Recruitment, retention, professional development, and performance evaluation of professors and staff are issues that educational institutions frequently face \cite{f12023}. These challenges may have an influence on the overall quality of education provided and the efficiency of the institution. Resolving these HR issues is essential to creating an effective environment for learning and encouraging continuous growth in educational institutions.
        
    	Same as other institutions, HRIS improves the HR operations and decision-making procedures in educational institutions. Educational institutions can improve data accuracy, expedite administrative procedures, and support strategic workforce planning through the use of HRIS \cite{edcpr12019}. With the use of HRIS, educational institutions may automate repetitive HR tasks, consolidate employees' data, and provide insightful data that helps with well-informed decision-making. The implementation of HRIS in educational environments enhances organizational performance overall, regulatory compliance, and operational efficiency.

    	The implementation of HRIS in educational institutions is an important initiative meant to improve organizational performance and modernize HR procedures. Technology is integrated to improve data management, expedite HR procedures, and increase communication within the organization when HRIS is used in educational settings \cite{s12020}. Educational institutions may analyze employee performance, allocate resources more efficiently, and match HR initiatives with academic objectives by utilizing HRIS. Better teaching and learning results, more staff satisfaction, and increased operational efficiency can all result from the effective adoption of HRIS in educational institutions.

    \subsection{Challenges in HRIS Implementation and Management}
        Organizations have many challenges in the implementation and management of Human Resources Information Systems (HRIS) when they adopt and maintain these systems. Employee resistance, unreasonable requirements, change management, the requirement for training, and setting up the proper IT infrastructure are common problems encountered during HRIS implementation \cite{kben12022}. These challenges must be carefully planned for and handled with strategic management in order to prevent HRIS from being implemented improperly. Research has indicated that resolving these issues is crucial to guaranteeing that HRIS is used in businesses in an efficient manner \cite{arc12020}.
        
        For businesses looking to streamline their HR procedures, maintenance and upgrade challenges in HRIS implementation and management present further challenges. Maintaining the operation and relevance of HRIS systems requires regular maintenance and timely upgrades \cite{ecaj12021}. Organizations frequently struggle with issues like user training, technical support, data quality maintenance, and system upgrades. To optimize the advantages of HRIS and guarantee its continuous efficacy in assisting HR activities, it is important to overcome this maintenance and upgrade barriers \cite{m12024}.

    \subsection{Trends in Human Resources Information Systems}
        Technological advancements have had an enormous impact on HR information systems, affecting how businesses manage their human resources and streamline HR procedures. HRIS is evolving into a more complex system with the integration of Artificial Intelligence, Machine Learning, and Deep Analytics, allowing predictive analytics for workforce planning, personalized learning and development plans, and improved recruitment tactics \cite{p12024}. These developments in technology are triggering a change in HR practices toward ones that are more data-driven and flexible, enabling firms to quickly respond to changing business requirements and make well-informed decisions.

        Enhancing user experience, using data analytics for decision-making, and integrating AI-driven solutions for automation and efficiency are the three main objectives of recent advances in Human Resources Information Systems (HRIS) \cite{s22024}. Cloud-based HRIS systems are becoming more and more popular among organizations because they provide real-time access to HR data, scalability, and flexibility. Furthermore, the incorporation of mobile HRIS applications facilitates employees' access to HR services at any time and location, hence fostering increased productivity and engagement. These patterns show a move toward more adaptable, user-focused HR procedures that make use of technology to boost business performance.
    
    \subsection{Redesign Considerations}
         One of the organizations that need to consider redesigning HRIS is to guarantee a smooth transition. The common reason for redesigning HRIS is because of the need for better operational efficiency, improved data accuracy, and to accommodate business requirements \cite{m12023}. By HRIS redesign, organizations can improve employee experiences, streamline workforce management procedures, and match HR operations with strategic goals. By identifying the reasons behind the redesign, organizations can establish clear goals and objectives for the HRIS transformation.

        The important aspects of HRIS redesign include assessing the limitations of the current system, determining the redesign's objectives, including major stakeholders, guaranteeing data security and compliance, and organizing change management \cite{is12020}. Important factors to take into account include comprehending the organization's particular HR requirements, assessing the new system's scalability and adaptability, and coordinating the HRIS redesign with organizational objectives. Furthermore, a successful redesign of HRIS involves carefully choosing the technology, guaranteeing seamless integration with existing systems, and offering users appropriate training and support \cite{ymqz12022}.
        
        Legacy HRIS has to be redesigned in order to take advantage of new features and capabilities according to the influence of the advancement of technology. Organizations are modernizing their HRIS to provide predictive analytics, customized HR services, and task automation via the amalgamation of Artificial Intelligence, Machine Learning, and data analytics \cite{mee12022}. The demand for a complete revamp of HRIS is being driven by technological improvements, which aim to increase operational efficiency, streamline labor management, and facilitate better decision-making. Organizations may foster innovation in HR processes, remain competitive, and adjust to shifting business environments by adopting new technology.
    
    \subsection{Migration Initiatives}
        According to Bakar, H. et.al.,  migration in the context of an information system is defined as the systematic transfer of data, applications, and processes from old systems to new platforms in order to improve operational effectiveness and meet changing business requirements \cite{hrd12022}. By reducing the risks connected with outdated technology, enhancing system performance, and guaranteeing compatibility with existing IT infrastructure, migration enables businesses to take advantage of cutting-edge innovations and maintain their edge in today's digital economy.
        
        Re-engineering, re-platforming, and re-hosting are a few common migration techniques that are tailored to the particular requirements of the company and the legacy system being migrated. \cite{hrd12020}. Re-platforming is the process of moving programs to a new platform without changing their essential functionality, whereas re-engineering is rewriting the system architecture to adjust to current requirements. Re-hosting, on the other hand, entails transferring the system to a different setting while preserving its present state. With the use of these techniques, companies may select the best course of action depending on variables like system complexity, financial resources, and time frame.
        
        In this migration process, it is important to consider having a meticulously planned migration strategy. In this way, it guarantees a seamless and effective migration from outdated systems to modern ones. Detailed planning of migration operations, determination of migration goals and objectives, risk assessment, stakeholder participation, and a complete assessment and analysis of the current system are all components of a well-planned migration strategy \cite{hrd12021}. Organizations may minimize operational disturbances, manage possible risks, and guarantee the smooth integration of new technologies into the current IT infrastructure by creating a comprehensive migration strategy. An organized method for planning migrations also makes it easier to manage schedules, allocate resources efficiently, and assess the process after migration to make sure everything goes as planned.


% ------------------------------------------------------------------ %
% ------Review of Related Systems----------------------------------- %
% ------------------------------------------------------------------ %

\section{Review of Related Systems}
    Within this chapter, the researchers will explore different pre-existing HRIS systems, their features, similarities, and relevance to addressing the identified problem. By examining pre-existing HRIS systems, researchers can identify unique advantages that can be leveraged to develop a more effective HR system solution. Furthermore, the chapter will explore the importance of Human Resource Information Systems (HRIS) in enhancing HR planning, job roles, performance reviews, and training initiatives that cover the multifaceted roles of HRIS in optimizing various HR procedures and sub-processes within organizations.
    
    \subsection{Overview of Popular HRIS Solutions}
    Throughout the years of HR systems, many different HRIS solutions have emerged, each offering unique features and capabilities tailored to organizational needs. Some allow services for generic HR web systems that are offering cloud-based services providing enhanced flexibility and accessibility compared to traditional data storage methods.

    Modern popular HRIS solutions offer a wide range of functionalities most often including the core common services such as:

    \begin{enumerate}
        \item \textbf{Human Resources Management:} This includes all employee/personnel-related management. Ranging from job management, performance evaluation, scheduling/time-related services e.g., attendance, leave, work schedule, etc.
        \item \textbf{Cloud-based Flexibility:} This allows for data storage or management through the use of cloud-based services. This allows for more scalable, reliable, and accessible data access due to its remoteness capability.
        \item \textbf{Time/Scheduling Services (e.g., Attendance, Leave, Calendar):} This process is used for employee attendance management and other scheduling processes e.g., work schedule management, faculty attendance, absent without official leave (AWOL), etc.
        \item \textbf{Payroll System:} This includes to timely compensation for employees with sub-processes including modules like automated payroll processing, tax compliance, benefits, reporting, wage management, employee self-service, etc.
    \end{enumerate}

    \subsection{Comparison of Features and Functionalities}
    In order to strengthen the proposed web application, a comparative analysis of different HRIS solutions needs to be gathered. This involves focusing on their strengths, weaknesses, and suitability for the University's requirements and use cases through different kinds of modules. For this, a comparison matrix of features and functionalities is established to visualize and clearly overview each system's strengths and the edge of the new proposed HR system over other pre-existing popular global systems.
    
    % Please add the following required packages to your document preamble:
% \usepackage{booktabs}
% \usepackage{multirow}
\begin{table}[H]
\centering
\resizebox{\textwidth}{!}
{\begin{tabular}{@{}llcccc@{}}
\toprule
                                                       & \textbf{Modules}                                    & \multicolumn{1}{l}{\textbf{ADNU HRIS}}             & \multicolumn{1}{l}{\textbf{OMNI HR}}               & \multicolumn{1}{l}{\textbf{DEEL HR}}               & \multicolumn{1}{l}{\textbf{PAYCOR HCM}}            \\ \midrule
\multicolumn{1}{c}{\multirow{19}{*}{\textbf{GENERIC}}} & Contacts                                            & \checkmark                          & \checkmark                          & \checkmark                          & \checkmark                          \\
\multicolumn{1}{c}{}                                   & Data Extraction                                     & \checkmark                          & \checkmark                          & \checkmark                          & \checkmark                          \\
\multicolumn{1}{c}{}                                   & Personal Info                                       & \checkmark                          & \checkmark                          & \checkmark                          & \checkmark                          \\
\multicolumn{1}{c}{}                                   & General Employee Status Tracker                     & \checkmark                          & \checkmark                          & \checkmark                          & \checkmark                          \\
\multicolumn{1}{c}{}                                   & Employee Profile                                    & \checkmark                          & \checkmark                          & \checkmark                          & \checkmark                          \\
\multicolumn{1}{c}{}                                   & Assignment Designation                              & \checkmark                          & \checkmark                          & \checkmark                          & \checkmark                          \\
\multicolumn{1}{c}{}                                   & Assignment Archive                                  & \checkmark                          & \checkmark                          & \checkmark                          & \checkmark                          \\
\multicolumn{1}{c}{}                                   & Faculty Rank                                        & \checkmark                          & \text{\ding{55}} & \text{\ding{55}} & \text{\ding{55}} \\
\multicolumn{1}{c}{}                                   & Academic Record                                     & \checkmark                          & \checkmark                          & \checkmark                          & \checkmark                          \\
\multicolumn{1}{c}{}                                   & Academic Awards                                     & \checkmark                          & \checkmark                          & \checkmark                          & \checkmark                          \\
\multicolumn{1}{c}{}                                   & Professional License Record                         & \checkmark                          & \checkmark                          & \checkmark                          & \checkmark                          \\
\multicolumn{1}{c}{}                                   & Training Attended Module                            & \checkmark                          & \checkmark                          & \checkmark                          & \checkmark                          \\
\multicolumn{1}{c}{}                                   & Performance Evaluation                              & \checkmark                          & \checkmark                          & \checkmark                          & \checkmark                          \\
\multicolumn{1}{c}{}                                   & COE                                                 & \checkmark                          & \text{\ding{55}} & \text{\ding{55}} & \text{\ding{55}} \\
\multicolumn{1}{c}{}                                   & COE Reports                                         & \checkmark                          & \text{\ding{55}} & \text{\ding{55}} & \text{\ding{55}} \\
\multicolumn{1}{c}{}                                   & Contracts/Appointment Generation Reports            & \checkmark                          & \checkmark                          & \checkmark                          & \checkmark                          \\
\multicolumn{1}{c}{}                                   & Health Record                                       & \checkmark                          & \checkmark                          & \checkmark                          & \checkmark                          \\
\multicolumn{1}{c}{}                                   & Cloud-based                                         & \text{\ding{55}} & \checkmark                          & \checkmark                          & \checkmark                          \\
\multicolumn{1}{c}{}                                   & Learning Management                                 & \text{\ding{55}} & \text{\ding{55}} & \text{\ding{55}} & \checkmark                          \\
\multirow{15}{*}{\textbf{TIMESYS}}                     & DTR Scanner                                         & \checkmark                          & \text{\ding{55}} & \text{\ding{55}} & \text{\ding{55}} \\
                                                       & Attendance Module                                   & \checkmark                          & \checkmark                          & \checkmark                          & \checkmark                          \\
                                                       & Attendance Archive                                  & \checkmark                          & \text{\ding{55}} & \text{\ding{55}} & \text{\ding{55}} \\
                                                       & Flexible Time Office                                & \checkmark                          & \checkmark                          & \checkmark                          & \checkmark                          \\
                                                       & Staff Attendance Report                             & \checkmark                          & \checkmark                          & \checkmark                          & \checkmark                          \\
                                                       & Calendar Management                                 & \checkmark                          & \text{\ding{55}} & \text{\ding{55}} & \checkmark                          \\
                                                       & Holiday Calendar Creation                           & \checkmark                          & \text{\ding{55}} & \text{\ding{55}} & \text{\ding{55}} \\
                                                       & Work Schedule Scheme                                & \checkmark                          & \checkmark                          & \checkmark                          & \checkmark                          \\
                                                       & Assign Work Schedule Scheme                         & \checkmark                          & \checkmark                          & \checkmark                          & \checkmark                          \\
                                                       & Work Schedule Scheme Checker                        & \checkmark                          & \checkmark                          & \checkmark                          & \checkmark                          \\
                                                       & Tardines Module                                     & \checkmark                          & \checkmark                          & \checkmark                          & \checkmark                          \\
                                                       & Remarks Module                                      & \checkmark                          & \checkmark                          & \checkmark                          & \checkmark                          \\
                                                       & AWOL Module                                         & \checkmark                          & \checkmark                          & \checkmark                          & \checkmark                          \\
                                                       & Overtime Module                                     & \checkmark                          & \checkmark                          & \checkmark                          & \checkmark                          \\
                                                       & Staff Monthly/Annual Attendance Report              & \checkmark                          & \text{\ding{55}} & \text{\ding{55}} & \text{\ding{55}} \\
\multirow{6}{*}{\textbf{LEAVESYS}}                     & Leave Application                                   & \checkmark                          & \checkmark                          & \checkmark                          & \checkmark                          \\
                                                       & Leave Reason                                        & \checkmark                          & \checkmark                          & \checkmark                          & \checkmark                          \\
                                                       & Leave Credits                                       & \checkmark                          & \checkmark                          & \checkmark                          & \checkmark                          \\
                                                       & Leave Credits Scheme                                & \checkmark                          & \checkmark                          & \checkmark                          & \checkmark                          \\
                                                       & Assign Leave Credit Scheme                          & \checkmark                          & \checkmark                          & \checkmark                          & \checkmark                          \\
                                                       & Process Leave Application with Leave Credits Report & \checkmark                          & \checkmark                          & \checkmark                          & \checkmark                          \\
\multirow{5}{*}{\textbf{FACSYS}}                       & Faculty Attendance                                  & \checkmark                          & \text{\ding{55}} & \text{\ding{55}} & \text{\ding{55}} \\
                                                       & Faculty Schedule                                    & \checkmark                          & \text{\ding{55}} & \text{\ding{55}} & \text{\ding{55}} \\
                                                       & Pending Faculty Schedule                            & \checkmark                          & \text{\ding{55}} & \text{\ding{55}} & \text{\ding{55}} \\
                                                       & Required Class Hours                                & \checkmark                          & \text{\ding{55}} & \text{\ding{55}} & \text{\ding{55}} \\
                                                       & Process Faculty Attendance Report                   & \checkmark                          & \text{\ding{55}} & \text{\ding{55}} & \text{\ding{55}} \\ \cmidrule(l){2-6} 
\end{tabular}}
\caption{ADNU HRIS in comparison with other HRM/HCM systems.}
\label{tab:my_label}
\end{table}
    
    \subsection{Evaluation of HRIS Solutions for Redesign}
    
    When organizations decide to redesign or migrate to a new HRIS system, it is crucial to critically evaluate the strengths and weaknesses of their current solution. This evaluation and analysis will help identify the areas that require improvement and set criteria for selecting the appropriate redesign approach. This section shall examine the evaluation process, highlighting the strengths and weaknesses of the current HRIS solution at the organization, and outlining the key criteria to consider when selecting a redesign strategy.
    
        \subsubsection{Strengths and Weaknesses of the current HRIS Solution}
        % Talk about the cons...
        According to the study, the current system struggles with maintaining a scalable system preventing other processes from progressing e.g., the payroll system. Not only that, the system lacks better efficiency in operations i.e., being able to instantly receive up-to-date information from other branches. Due to the system's age in the technology stack, the ability for further updates and technological advancement to the application is hindered and is stuck to not break the operation e.g., the DTR scanner in HR uses an old version and limited compatibility with other software only accessible to Internet Explorer (IE). 

        % talk about the pros...
        Despite its issues, the system still performs as functional and operational as the current application for handling the HRIS and has well-established processes that define most of the core modules of the University's requirements. 
        
        \subsubsection{Criteria for Selecting an HRIS Redesign}
        To solve the HRIS's issues, criteria for redesigning the HRIS have to be set, and must consider various factors before creating an application redesign. These factors must be included:

        \begin{itemize}
            \item[] \textbf{Scalability:} By allowing the system to be as scalable as possible, organizations can effectively manage growth and adapt to changing business needs. This involves creating a database design that can adapt to various use cases and is generic. 
            \item[] \textbf{Ease-of-Use:} This involves developing and improving user experience (UX) by creating a responsive and modernized user interface (UI) for employees and admins to use. Doing this will improve not only the ease of access but will ensure better communication and exchange of information from other departments.
            \item[] \textbf{Modernized Frameworks and Technology Stack:} This involves replacing the existing technology stack with modern frameworks and tools to develop an up-to-date HRIS system. With this, the developers can use vast web technologies for better compatibility, and various design possibilities to create a better system accessible not only within University grounds but through internal networks.
            \item[] \textbf{Integration with UIIS:} This involves centralizing the database through integration with the UIIS. This allows for improved efficiency due to its integration not only to the UIIS but to the MIS; aligning with the goals of the MIS office which is to create unified and interconnected systems within the University to streamline operations and workflows. This way data redundancy and inconsistencies eliminated.
        \end{itemize}
    
    \subsection{Case Study Analysis of HRIS Redesign and Migration}
    To further prove the study, deeper analysis and examination of other works in HRIS redesign has to be considered in support of claiming the need for an HRIS redesign in the first place; why and how implementing HRIS equates to effectiveness in an organization's operations.
    
        \subsubsection{Case Studies of Successful HRIS Migration}
        Among millions of organizations across the world, research would show that at least 55\% of organizations are only using Applicant Tracking System (ATS) or HRIS, and 45\% of organizations currently do not use ATS or HRIS \cite{ms12019}. This meant that at least a large number of organizations have personal information management as decentralized; leading to issues like data inconsistency, redundancy, and inaccuracy.

        In a recent study PT Pertamina based in India, conducted research regarding the effectiveness of their system --- Information About Me Human Resource Information System (I-AM HRIS). According to the article, PT Pertamina developed the I-AM HRIS application in 2016 to facilitate HR administration processes and bring "One Pertamina" uniformity. The study used the DeLone and McLean IS Success model along with Quality Function Deployment (QFD) analysis to evaluate I-AM HRIS implementation effectiveness. However, the system became ineffective, with a survey found that 67.8\% of employees were unaware of the service facilities provided by I-AM HRIS, 52.4\% did not frequently use the application, and employee satisfaction was not achieved \cite{rh12021}. 

        With this analysis, the study recommended redesigning I-AM HRIS to improve quality and achieve better employee service satisfaction, especially focusing on increasing process speed and ease of use.
        
        \subsubsection{Challenges and Solutions}
        The problem with redesigning and migrating to a new system despite being just the same existing system but redesigned, is that it raises potential obstacles and challenges for developers to make. One of the common issues that organizations will likely resist in this proposal is the idea of change from employees to use the newer processes. More often than not, some organizations and people are used to legacy processes and will resist any changes. With this, comes considerations for the company to conduct training and knowledge transfer gaps. In addition, budgetary and resource constraints for the redesign and migration project may become a problem for some organizations.

        Another critical issue is the data migration and integration across multiple systems, migrating to a newer system meant a probable change in the database schema. This major change will likely break the records and hinder the migration process. Hence, the emphasis on creating scalable and dynamic schemas is to be practiced as they serve as the foundation at the start of the design.
        
        Despite these challenges, potential solutions for handling them can be through effective change management and communication strategies as well as conducting a comprehensive training program tailored to different user types. The team can also establish a project management strategy through iterative prototyping and user feedback loops to achieve user requirements. Moreover, create robust data migration planning and quality assurance before implementation.
        
        \subsubsection{Outcomes and Benefits Achieved Post-Migration}
        While the paper did not actually achieve post-migration and focused more on the evaluation of the existing state of the I-AM HRIS implementation using gap analysis and QFD, they still provided recommendations for redesigning I-AM HRIS to improve quality and achieve better employee service satisfaction, especially focusing on increasing process speed and ease-of-use. With this, we can infer potential outcomes and benefits earned if the migration were successful. 

        One would be the improved operational efficiencies in HR processes like employee data management, payroll, recruitment, etc. through automation and integration. Better scalable systems come with better data quality, consistency, and accuracy by redesigning the system to meet employee expectations allowing an increase in employee satisfaction and productivity by addressing gaps like slow processes mentioned in the case study and the lack of user-friendliness identified within the analysis.

        Another benefit can be cost savings by reducing manual efforts and errors through the redesigned HRIS capabilities and overall better decision-making through access to comprehensive, real-time workforce analytics and reporting from the HRIS.

        With this case study, the researchers can draw out similarities within the ADNU HRIS' current system being in the similar state with it being limited, lacking in accessibility, user-friendliness, and compatibility; that are in need of a new redesign and migration. Doing so will achieve the proposed objectives of this study and enhance the HR operations within the University.